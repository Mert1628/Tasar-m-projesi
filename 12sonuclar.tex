\section{SONUÇLAR VE ÖNERİLER}
\subsection{Sonuçlar}

    Bu proje, ultrasonik sensör, servo motor, LCD ve buzzer gibi bileşenleri kullanarak otomatik park sisteminin etkin bir prototipini sunmuştur. Özellikle mesafe algılama ve kullanıcıya görsel-işitsel geri bildirim sağlama konusunda başarılı bir performans sergilemiştir.

    Sistem, farklı mesafelerde doğru çalışmakta ve sürücülerin park işlemini kolaylaştırmaktadır.

\subsection{Öneriler}
    Bu öneriler, sistemin hem performans hem de kullanıcı deneyimi açısından geliştirilmesine yönelik bir yol haritası sunmaktadır.
\begin{itemize}
\item \textbf{\underline{Sensör Hassasiyeti:}}  Daha hassas ve geniş algılama aralığına sahip sensörler kullanılarak sistem geliştirilebilir.

\item \textbf{\underline{Kablosuz Bağlantı:}} Bluetooth veya Wi-Fi modülleri eklenerek uzaktan kontrol ve gözetim sağlanabilir.

\item \textbf{\underline{Gelişmiş Uyarı Mekanizmaları:}}  LED ekran veya akıllı telefon uygulamaları ile entegre edilerek daha detaylı bilgi sunulabilir.

\item \textbf{\underline{Ek Sensörler:}} Sisteme eklenen kamera veya lidar sensörler ile park işlemi daha akıllı hale getirilebilir.

\item \textbf{\underline{Enerji Verimliliği:}} Sistem, enerji tasarrufu sağlamak için uyku modları veya daha verimli enerji kaynakları ile donatılabilir.
\end{itemize}

    Projenin gelecekteki uygulamaları, akıllı aracın park etme sistemlerinde daha fazla otomasyon ve kullanım kolaylığı sağlayacak şekilde geliştirilmesini sağlayabilir.

