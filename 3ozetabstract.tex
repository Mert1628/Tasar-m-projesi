\centerline{\bf ÖZET}
\addcontentsline{toc}{section}{ÖZET}
\begin{center}
\textbf{Projenin Amacı}
\end{center} 
 \begin{minipage}{\linewidth}
     
     Araç park etme işlemini kolaylaştırmak ve daha güvenli hale getirmek için Arduino tabanlı bir otomatik park sistemi geliştirmektir. Sistem, HC-SR04 ultrasonik sensör kullanarak aracın sağ tarafındaki mesafeleri ölçer ve bu verilere dayanarak servo motor ile park işlemini gerçekleştirir. Ayrıca, kullanıcıyı buzzer ile sesli uyarılar ve LCD ekran ile görsel bilgiler sağlayarak yönlendirir. Bu proje, dar alanlarda güvenli park etme sorunlarına etkili bir çözüm sunarken, kullanıcıların park sürecini daha hızlı ve kolay bir şekilde tamamlamalarına olanak tanır.
     
\end{minipage}
\vspace{1cm}
\begin{center}
\textbf{Projenin Kapsamı}
\end{center}
  \begin{minipage}{\linewidth}
     
     Proje kapsamında, araç park etme sürecini otomatikleştirmek ve kullanıcıya kolaylık sağlamak için Arduino platformunu kullanarak bir mesafe ölçüm ve park sistemi geliştirilmesiyle sınırlıdır. Sistem, HC-SR04 ultrasonik sensör ile mesafe ölçümü yapar, servo motor ile aracın hareketini kontrol eder ve buzzer ile sesli uyarılar verirken LCD ekran üzerinden görsel bilgi sağlar. Proje, küçük ölçekli araç modelleri üzerinde test edilerek, gerçek dünyadaki park sorunlarına yönelik bir prototip sunmayı amaçlamaktadır. Sistemin donanım ve yazılım bileşenleri, temel park etme senaryolarını gerçekleştirecek şekilde tasarlanmış olup, karmaşık trafik ortamları veya büyük ölçekli araç uygulamaları bu projenin kapsamına dahil edilmemiştir.
     
 \end{minipage}
\vspace{1cm}
\begin{center}
\textbf{Sonuçlar}
\end{center}
\begin{minipage}{\linewidth}
    
    Bu proje, Arduino tabanlı bir otomatik park sistemi geliştirilerek mesafe ölçümünün ve park işlemlerinin başarılı bir şekilde gerçekleştirilebileceğini göstermiştir. Sistem, kullanıcıya güvenli park desteği sağlarken görsel ve işitsel geri bildirim sunmuştur. Gelecekte daha gelişmiş sensörler ve algoritmalarla performans artırılabilir.
    
\end{minipage}
%%%%%%%%%%%%%%%%%%%%%%%%%%%%%%%%%%%%%%%ABSTRACT%%%%%%%%%%%%%%%%%%%%%%%%%%%%%%%%%%%%%%%%%%
\newpage
\centerline{\bf ABSTRACT}
\addcontentsline{toc}{section}{ABSTRACT}
\begin{center}
\textbf{Project Objective}
\end{center}

\begin{minipage}{\linewidth}
    
    The aim of this project is to develop an Arduino based automatic parking system to make vehicle parking easier and safer. The system measures the distances on the right side of the vehicle using the HC-SR04 ultrasonic sensor and performs the parking process with the servo motor based on this data. It also guides the user by providing audible warnings with buzzer and visual information with LCD screen. This project provides an effective solution to the problems of safe parking in tight spaces, while allowing users to complete the parking process more quickly and easily.


\end{minipage}
\vspace{1cm}
\begin{center}
\textbf{Scope of Project}
\end{center}

 \begin{minipage}{\linewidth}
    
   The scope of the project is limited to the development of a distance measurement and parking system using the Arduino platform to automate the vehicle parking process and provide convenience to the user. The system measures distance with the HC-SR04 ultrasonic sensor, controls the movement of the vehicle with the servo motor and provides visual information on the LCD screen while giving audible warnings with the buzzer. The project aims to provide a prototype for real-world parking problems by testing on small-scale vehicle models. The hardware and software components of the system are designed to perform basic parking scenarios, and complex traffic environments or large-scale vehicle applications are not included in the scope of this project.

\end{minipage}
\vspace{1cm}
\begin{center}
\textbf{Results}
\end{center}

\begin{minipage}{\linewidth}
    
    This project demonstrated that distance measurement and parking operations can be performed successfully by developing an Arduino based automatic parking system. The system provided visual and auditory feedback while providing safe parking support to the user. In the future, performance can be improved with more advanced sensors and algorithms.
\end{minipage}